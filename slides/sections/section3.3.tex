\section{顺反组}
\subsection{概述}
\begin{frame}[label=current]
  \frametitle{顺反组}
  \begin{block}{维基百科}
顺反组(cistrome)指的是“全基因组尺度下反式作用因子的顺式作用靶点的集合,也可以说是在体情况下转录因子结合位点或组蛋白修饰在全基因组上的位置”。“顺反组”这一术语是cistron(顺反子)和genome(基因组)的混成词,最初由达纳-法伯癌症研究所和哈佛医学院的研究者命名。\\
\vspace{0.3em}
    染色质免疫沉淀等技术结合微阵列分析“ChIP-on-chip”或大规模并行DNA测序“ChIP-Seq”极大地方便了对转录因子及其它染色质相关蛋白的顺反组的定义。
  \end{block}
  \pause
  \begin{block}{百度百科}
    顺反组(cistrome)是由“Dana-Farber癌症研究所”与哈佛医学院的科学家提出的遗传学术语,用于定义一个反式(trans)调控因子在基因组(genome)范围内的作用对象——一组顺式(cis)作用元素。一些技术,例如免疫共沉淀与基因芯片结合的技术(ChIP-on-chip),已经被广泛的应用于发现转录因子以及其他染色质相关因子的顺反组。
  \end{block}
\end{frame}

\subsection{ChIP-Seq}
\subsubsection{技术简介}
\subsubsection{数据分析}

